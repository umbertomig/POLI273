\documentclass[11pt]{article}

    \usepackage[breakable]{tcolorbox}
    \usepackage{parskip} % Stop auto-indenting (to mimic markdown behaviour)
    

    % Basic figure setup, for now with no caption control since it's done
    % automatically by Pandoc (which extracts ![](path) syntax from Markdown).
    \usepackage{graphicx}
    % Maintain compatibility with old templates. Remove in nbconvert 6.0
    \let\Oldincludegraphics\includegraphics
    % Ensure that by default, figures have no caption (until we provide a
    % proper Figure object with a Caption API and a way to capture that
    % in the conversion process - todo).
    \usepackage{caption}
    \DeclareCaptionFormat{nocaption}{}
    \captionsetup{format=nocaption,aboveskip=0pt,belowskip=0pt}

    \usepackage{float}
    \floatplacement{figure}{H} % forces figures to be placed at the correct location
    \usepackage{xcolor} % Allow colors to be defined
    \usepackage{enumerate} % Needed for markdown enumerations to work
    \usepackage{geometry} % Used to adjust the document margins
    \usepackage{amsmath} % Equations
    \usepackage{amssymb} % Equations
    \usepackage{textcomp} % defines textquotesingle
    % Hack from http://tex.stackexchange.com/a/47451/13684:
    \AtBeginDocument{%
        \def\PYZsq{\textquotesingle}% Upright quotes in Pygmentized code
    }
    \usepackage{upquote} % Upright quotes for verbatim code
    \usepackage{eurosym} % defines \euro

    \usepackage{iftex}
    \ifPDFTeX
        \usepackage[T1]{fontenc}
        \IfFileExists{alphabeta.sty}{
              \usepackage{alphabeta}
          }{
              \usepackage[mathletters]{ucs}
              \usepackage[utf8x]{inputenc}
          }
    \else
        \usepackage{fontspec}
        \usepackage{unicode-math}
    \fi

    \usepackage{fancyvrb} % verbatim replacement that allows latex
    \usepackage{grffile} % extends the file name processing of package graphics
                         % to support a larger range
    \makeatletter % fix for old versions of grffile with XeLaTeX
    \@ifpackagelater{grffile}{2019/11/01}
    {
      % Do nothing on new versions
    }
    {
      \def\Gread@@xetex#1{%
        \IfFileExists{"\Gin@base".bb}%
        {\Gread@eps{\Gin@base.bb}}%
        {\Gread@@xetex@aux#1}%
      }
    }
    \makeatother
    \usepackage[Export]{adjustbox} % Used to constrain images to a maximum size
    \adjustboxset{max size={0.9\linewidth}{0.9\paperheight}}

    % The hyperref package gives us a pdf with properly built
    % internal navigation ('pdf bookmarks' for the table of contents,
    % internal cross-reference links, web links for URLs, etc.)
    \usepackage{hyperref}
    % The default LaTeX title has an obnoxious amount of whitespace. By default,
    % titling removes some of it. It also provides customization options.
    \usepackage{titling}
    \usepackage{longtable} % longtable support required by pandoc >1.10
    \usepackage{booktabs}  % table support for pandoc > 1.12.2
    \usepackage{array}     % table support for pandoc >= 2.11.3
    \usepackage{calc}      % table minipage width calculation for pandoc >= 2.11.1
    \usepackage[inline]{enumitem} % IRkernel/repr support (it uses the enumerate* environment)
    \usepackage[normalem]{ulem} % ulem is needed to support strikethroughs (\sout)
                                % normalem makes italics be italics, not underlines
    \usepackage{mathrsfs}
    

    
    % Colors for the hyperref package
    \definecolor{urlcolor}{rgb}{0,.145,.698}
    \definecolor{linkcolor}{rgb}{.71,0.21,0.01}
    \definecolor{citecolor}{rgb}{.12,.54,.11}

    % ANSI colors
    \definecolor{ansi-black}{HTML}{3E424D}
    \definecolor{ansi-black-intense}{HTML}{282C36}
    \definecolor{ansi-red}{HTML}{E75C58}
    \definecolor{ansi-red-intense}{HTML}{B22B31}
    \definecolor{ansi-green}{HTML}{00A250}
    \definecolor{ansi-green-intense}{HTML}{007427}
    \definecolor{ansi-yellow}{HTML}{DDB62B}
    \definecolor{ansi-yellow-intense}{HTML}{B27D12}
    \definecolor{ansi-blue}{HTML}{208FFB}
    \definecolor{ansi-blue-intense}{HTML}{0065CA}
    \definecolor{ansi-magenta}{HTML}{D160C4}
    \definecolor{ansi-magenta-intense}{HTML}{A03196}
    \definecolor{ansi-cyan}{HTML}{60C6C8}
    \definecolor{ansi-cyan-intense}{HTML}{258F8F}
    \definecolor{ansi-white}{HTML}{C5C1B4}
    \definecolor{ansi-white-intense}{HTML}{A1A6B2}
    \definecolor{ansi-default-inverse-fg}{HTML}{FFFFFF}
    \definecolor{ansi-default-inverse-bg}{HTML}{000000}

    % common color for the border for error outputs.
    \definecolor{outerrorbackground}{HTML}{FFDFDF}

    % commands and environments needed by pandoc snippets
    % extracted from the output of `pandoc -s`
    \providecommand{\tightlist}{%
      \setlength{\itemsep}{0pt}\setlength{\parskip}{0pt}}
    \DefineVerbatimEnvironment{Highlighting}{Verbatim}{commandchars=\\\{\}}
    % Add ',fontsize=\small' for more characters per line
    \newenvironment{Shaded}{}{}
    \newcommand{\KeywordTok}[1]{\textcolor[rgb]{0.00,0.44,0.13}{\textbf{{#1}}}}
    \newcommand{\DataTypeTok}[1]{\textcolor[rgb]{0.56,0.13,0.00}{{#1}}}
    \newcommand{\DecValTok}[1]{\textcolor[rgb]{0.25,0.63,0.44}{{#1}}}
    \newcommand{\BaseNTok}[1]{\textcolor[rgb]{0.25,0.63,0.44}{{#1}}}
    \newcommand{\FloatTok}[1]{\textcolor[rgb]{0.25,0.63,0.44}{{#1}}}
    \newcommand{\CharTok}[1]{\textcolor[rgb]{0.25,0.44,0.63}{{#1}}}
    \newcommand{\StringTok}[1]{\textcolor[rgb]{0.25,0.44,0.63}{{#1}}}
    \newcommand{\CommentTok}[1]{\textcolor[rgb]{0.38,0.63,0.69}{\textit{{#1}}}}
    \newcommand{\OtherTok}[1]{\textcolor[rgb]{0.00,0.44,0.13}{{#1}}}
    \newcommand{\AlertTok}[1]{\textcolor[rgb]{1.00,0.00,0.00}{\textbf{{#1}}}}
    \newcommand{\FunctionTok}[1]{\textcolor[rgb]{0.02,0.16,0.49}{{#1}}}
    \newcommand{\RegionMarkerTok}[1]{{#1}}
    \newcommand{\ErrorTok}[1]{\textcolor[rgb]{1.00,0.00,0.00}{\textbf{{#1}}}}
    \newcommand{\NormalTok}[1]{{#1}}

    % Additional commands for more recent versions of Pandoc
    \newcommand{\ConstantTok}[1]{\textcolor[rgb]{0.53,0.00,0.00}{{#1}}}
    \newcommand{\SpecialCharTok}[1]{\textcolor[rgb]{0.25,0.44,0.63}{{#1}}}
    \newcommand{\VerbatimStringTok}[1]{\textcolor[rgb]{0.25,0.44,0.63}{{#1}}}
    \newcommand{\SpecialStringTok}[1]{\textcolor[rgb]{0.73,0.40,0.53}{{#1}}}
    \newcommand{\ImportTok}[1]{{#1}}
    \newcommand{\DocumentationTok}[1]{\textcolor[rgb]{0.73,0.13,0.13}{\textit{{#1}}}}
    \newcommand{\AnnotationTok}[1]{\textcolor[rgb]{0.38,0.63,0.69}{\textbf{\textit{{#1}}}}}
    \newcommand{\CommentVarTok}[1]{\textcolor[rgb]{0.38,0.63,0.69}{\textbf{\textit{{#1}}}}}
    \newcommand{\VariableTok}[1]{\textcolor[rgb]{0.10,0.09,0.49}{{#1}}}
    \newcommand{\ControlFlowTok}[1]{\textcolor[rgb]{0.00,0.44,0.13}{\textbf{{#1}}}}
    \newcommand{\OperatorTok}[1]{\textcolor[rgb]{0.40,0.40,0.40}{{#1}}}
    \newcommand{\BuiltInTok}[1]{{#1}}
    \newcommand{\ExtensionTok}[1]{{#1}}
    \newcommand{\PreprocessorTok}[1]{\textcolor[rgb]{0.74,0.48,0.00}{{#1}}}
    \newcommand{\AttributeTok}[1]{\textcolor[rgb]{0.49,0.56,0.16}{{#1}}}
    \newcommand{\InformationTok}[1]{\textcolor[rgb]{0.38,0.63,0.69}{\textbf{\textit{{#1}}}}}
    \newcommand{\WarningTok}[1]{\textcolor[rgb]{0.38,0.63,0.69}{\textbf{\textit{{#1}}}}}


    % Define a nice break command that doesn't care if a line doesn't already
    % exist.
    \def\br{\hspace*{\fill} \\* }
    % Math Jax compatibility definitions
    \def\gt{>}
    \def\lt{<}
    \let\Oldtex\TeX
    \let\Oldlatex\LaTeX
    \renewcommand{\TeX}{\textrm{\Oldtex}}
    \renewcommand{\LaTeX}{\textrm{\Oldlatex}}
    % Document parameters
    % Document title
    \title{Lecture06}
    
    
    
    
    
% Pygments definitions
\makeatletter
\def\PY@reset{\let\PY@it=\relax \let\PY@bf=\relax%
    \let\PY@ul=\relax \let\PY@tc=\relax%
    \let\PY@bc=\relax \let\PY@ff=\relax}
\def\PY@tok#1{\csname PY@tok@#1\endcsname}
\def\PY@toks#1+{\ifx\relax#1\empty\else%
    \PY@tok{#1}\expandafter\PY@toks\fi}
\def\PY@do#1{\PY@bc{\PY@tc{\PY@ul{%
    \PY@it{\PY@bf{\PY@ff{#1}}}}}}}
\def\PY#1#2{\PY@reset\PY@toks#1+\relax+\PY@do{#2}}

\@namedef{PY@tok@w}{\def\PY@tc##1{\textcolor[rgb]{0.73,0.73,0.73}{##1}}}
\@namedef{PY@tok@c}{\let\PY@it=\textit\def\PY@tc##1{\textcolor[rgb]{0.24,0.48,0.48}{##1}}}
\@namedef{PY@tok@cp}{\def\PY@tc##1{\textcolor[rgb]{0.61,0.40,0.00}{##1}}}
\@namedef{PY@tok@k}{\let\PY@bf=\textbf\def\PY@tc##1{\textcolor[rgb]{0.00,0.50,0.00}{##1}}}
\@namedef{PY@tok@kp}{\def\PY@tc##1{\textcolor[rgb]{0.00,0.50,0.00}{##1}}}
\@namedef{PY@tok@kt}{\def\PY@tc##1{\textcolor[rgb]{0.69,0.00,0.25}{##1}}}
\@namedef{PY@tok@o}{\def\PY@tc##1{\textcolor[rgb]{0.40,0.40,0.40}{##1}}}
\@namedef{PY@tok@ow}{\let\PY@bf=\textbf\def\PY@tc##1{\textcolor[rgb]{0.67,0.13,1.00}{##1}}}
\@namedef{PY@tok@nb}{\def\PY@tc##1{\textcolor[rgb]{0.00,0.50,0.00}{##1}}}
\@namedef{PY@tok@nf}{\def\PY@tc##1{\textcolor[rgb]{0.00,0.00,1.00}{##1}}}
\@namedef{PY@tok@nc}{\let\PY@bf=\textbf\def\PY@tc##1{\textcolor[rgb]{0.00,0.00,1.00}{##1}}}
\@namedef{PY@tok@nn}{\let\PY@bf=\textbf\def\PY@tc##1{\textcolor[rgb]{0.00,0.00,1.00}{##1}}}
\@namedef{PY@tok@ne}{\let\PY@bf=\textbf\def\PY@tc##1{\textcolor[rgb]{0.80,0.25,0.22}{##1}}}
\@namedef{PY@tok@nv}{\def\PY@tc##1{\textcolor[rgb]{0.10,0.09,0.49}{##1}}}
\@namedef{PY@tok@no}{\def\PY@tc##1{\textcolor[rgb]{0.53,0.00,0.00}{##1}}}
\@namedef{PY@tok@nl}{\def\PY@tc##1{\textcolor[rgb]{0.46,0.46,0.00}{##1}}}
\@namedef{PY@tok@ni}{\let\PY@bf=\textbf\def\PY@tc##1{\textcolor[rgb]{0.44,0.44,0.44}{##1}}}
\@namedef{PY@tok@na}{\def\PY@tc##1{\textcolor[rgb]{0.41,0.47,0.13}{##1}}}
\@namedef{PY@tok@nt}{\let\PY@bf=\textbf\def\PY@tc##1{\textcolor[rgb]{0.00,0.50,0.00}{##1}}}
\@namedef{PY@tok@nd}{\def\PY@tc##1{\textcolor[rgb]{0.67,0.13,1.00}{##1}}}
\@namedef{PY@tok@s}{\def\PY@tc##1{\textcolor[rgb]{0.73,0.13,0.13}{##1}}}
\@namedef{PY@tok@sd}{\let\PY@it=\textit\def\PY@tc##1{\textcolor[rgb]{0.73,0.13,0.13}{##1}}}
\@namedef{PY@tok@si}{\let\PY@bf=\textbf\def\PY@tc##1{\textcolor[rgb]{0.64,0.35,0.47}{##1}}}
\@namedef{PY@tok@se}{\let\PY@bf=\textbf\def\PY@tc##1{\textcolor[rgb]{0.67,0.36,0.12}{##1}}}
\@namedef{PY@tok@sr}{\def\PY@tc##1{\textcolor[rgb]{0.64,0.35,0.47}{##1}}}
\@namedef{PY@tok@ss}{\def\PY@tc##1{\textcolor[rgb]{0.10,0.09,0.49}{##1}}}
\@namedef{PY@tok@sx}{\def\PY@tc##1{\textcolor[rgb]{0.00,0.50,0.00}{##1}}}
\@namedef{PY@tok@m}{\def\PY@tc##1{\textcolor[rgb]{0.40,0.40,0.40}{##1}}}
\@namedef{PY@tok@gh}{\let\PY@bf=\textbf\def\PY@tc##1{\textcolor[rgb]{0.00,0.00,0.50}{##1}}}
\@namedef{PY@tok@gu}{\let\PY@bf=\textbf\def\PY@tc##1{\textcolor[rgb]{0.50,0.00,0.50}{##1}}}
\@namedef{PY@tok@gd}{\def\PY@tc##1{\textcolor[rgb]{0.63,0.00,0.00}{##1}}}
\@namedef{PY@tok@gi}{\def\PY@tc##1{\textcolor[rgb]{0.00,0.52,0.00}{##1}}}
\@namedef{PY@tok@gr}{\def\PY@tc##1{\textcolor[rgb]{0.89,0.00,0.00}{##1}}}
\@namedef{PY@tok@ge}{\let\PY@it=\textit}
\@namedef{PY@tok@gs}{\let\PY@bf=\textbf}
\@namedef{PY@tok@gp}{\let\PY@bf=\textbf\def\PY@tc##1{\textcolor[rgb]{0.00,0.00,0.50}{##1}}}
\@namedef{PY@tok@go}{\def\PY@tc##1{\textcolor[rgb]{0.44,0.44,0.44}{##1}}}
\@namedef{PY@tok@gt}{\def\PY@tc##1{\textcolor[rgb]{0.00,0.27,0.87}{##1}}}
\@namedef{PY@tok@err}{\def\PY@bc##1{{\setlength{\fboxsep}{\string -\fboxrule}\fcolorbox[rgb]{1.00,0.00,0.00}{1,1,1}{\strut ##1}}}}
\@namedef{PY@tok@kc}{\let\PY@bf=\textbf\def\PY@tc##1{\textcolor[rgb]{0.00,0.50,0.00}{##1}}}
\@namedef{PY@tok@kd}{\let\PY@bf=\textbf\def\PY@tc##1{\textcolor[rgb]{0.00,0.50,0.00}{##1}}}
\@namedef{PY@tok@kn}{\let\PY@bf=\textbf\def\PY@tc##1{\textcolor[rgb]{0.00,0.50,0.00}{##1}}}
\@namedef{PY@tok@kr}{\let\PY@bf=\textbf\def\PY@tc##1{\textcolor[rgb]{0.00,0.50,0.00}{##1}}}
\@namedef{PY@tok@bp}{\def\PY@tc##1{\textcolor[rgb]{0.00,0.50,0.00}{##1}}}
\@namedef{PY@tok@fm}{\def\PY@tc##1{\textcolor[rgb]{0.00,0.00,1.00}{##1}}}
\@namedef{PY@tok@vc}{\def\PY@tc##1{\textcolor[rgb]{0.10,0.09,0.49}{##1}}}
\@namedef{PY@tok@vg}{\def\PY@tc##1{\textcolor[rgb]{0.10,0.09,0.49}{##1}}}
\@namedef{PY@tok@vi}{\def\PY@tc##1{\textcolor[rgb]{0.10,0.09,0.49}{##1}}}
\@namedef{PY@tok@vm}{\def\PY@tc##1{\textcolor[rgb]{0.10,0.09,0.49}{##1}}}
\@namedef{PY@tok@sa}{\def\PY@tc##1{\textcolor[rgb]{0.73,0.13,0.13}{##1}}}
\@namedef{PY@tok@sb}{\def\PY@tc##1{\textcolor[rgb]{0.73,0.13,0.13}{##1}}}
\@namedef{PY@tok@sc}{\def\PY@tc##1{\textcolor[rgb]{0.73,0.13,0.13}{##1}}}
\@namedef{PY@tok@dl}{\def\PY@tc##1{\textcolor[rgb]{0.73,0.13,0.13}{##1}}}
\@namedef{PY@tok@s2}{\def\PY@tc##1{\textcolor[rgb]{0.73,0.13,0.13}{##1}}}
\@namedef{PY@tok@sh}{\def\PY@tc##1{\textcolor[rgb]{0.73,0.13,0.13}{##1}}}
\@namedef{PY@tok@s1}{\def\PY@tc##1{\textcolor[rgb]{0.73,0.13,0.13}{##1}}}
\@namedef{PY@tok@mb}{\def\PY@tc##1{\textcolor[rgb]{0.40,0.40,0.40}{##1}}}
\@namedef{PY@tok@mf}{\def\PY@tc##1{\textcolor[rgb]{0.40,0.40,0.40}{##1}}}
\@namedef{PY@tok@mh}{\def\PY@tc##1{\textcolor[rgb]{0.40,0.40,0.40}{##1}}}
\@namedef{PY@tok@mi}{\def\PY@tc##1{\textcolor[rgb]{0.40,0.40,0.40}{##1}}}
\@namedef{PY@tok@il}{\def\PY@tc##1{\textcolor[rgb]{0.40,0.40,0.40}{##1}}}
\@namedef{PY@tok@mo}{\def\PY@tc##1{\textcolor[rgb]{0.40,0.40,0.40}{##1}}}
\@namedef{PY@tok@ch}{\let\PY@it=\textit\def\PY@tc##1{\textcolor[rgb]{0.24,0.48,0.48}{##1}}}
\@namedef{PY@tok@cm}{\let\PY@it=\textit\def\PY@tc##1{\textcolor[rgb]{0.24,0.48,0.48}{##1}}}
\@namedef{PY@tok@cpf}{\let\PY@it=\textit\def\PY@tc##1{\textcolor[rgb]{0.24,0.48,0.48}{##1}}}
\@namedef{PY@tok@c1}{\let\PY@it=\textit\def\PY@tc##1{\textcolor[rgb]{0.24,0.48,0.48}{##1}}}
\@namedef{PY@tok@cs}{\let\PY@it=\textit\def\PY@tc##1{\textcolor[rgb]{0.24,0.48,0.48}{##1}}}

\def\PYZbs{\char`\\}
\def\PYZus{\char`\_}
\def\PYZob{\char`\{}
\def\PYZcb{\char`\}}
\def\PYZca{\char`\^}
\def\PYZam{\char`\&}
\def\PYZlt{\char`\<}
\def\PYZgt{\char`\>}
\def\PYZsh{\char`\#}
\def\PYZpc{\char`\%}
\def\PYZdl{\char`\$}
\def\PYZhy{\char`\-}
\def\PYZsq{\char`\'}
\def\PYZdq{\char`\"}
\def\PYZti{\char`\~}
% for compatibility with earlier versions
\def\PYZat{@}
\def\PYZlb{[}
\def\PYZrb{]}
\makeatother


    % For linebreaks inside Verbatim environment from package fancyvrb.
    \makeatletter
        \newbox\Wrappedcontinuationbox
        \newbox\Wrappedvisiblespacebox
        \newcommand*\Wrappedvisiblespace {\textcolor{red}{\textvisiblespace}}
        \newcommand*\Wrappedcontinuationsymbol {\textcolor{red}{\llap{\tiny$\m@th\hookrightarrow$}}}
        \newcommand*\Wrappedcontinuationindent {3ex }
        \newcommand*\Wrappedafterbreak {\kern\Wrappedcontinuationindent\copy\Wrappedcontinuationbox}
        % Take advantage of the already applied Pygments mark-up to insert
        % potential linebreaks for TeX processing.
        %        {, <, #, %, $, ' and ": go to next line.
        %        _, }, ^, &, >, - and ~: stay at end of broken line.
        % Use of \textquotesingle for straight quote.
        \newcommand*\Wrappedbreaksatspecials {%
            \def\PYGZus{\discretionary{\char`\_}{\Wrappedafterbreak}{\char`\_}}%
            \def\PYGZob{\discretionary{}{\Wrappedafterbreak\char`\{}{\char`\{}}%
            \def\PYGZcb{\discretionary{\char`\}}{\Wrappedafterbreak}{\char`\}}}%
            \def\PYGZca{\discretionary{\char`\^}{\Wrappedafterbreak}{\char`\^}}%
            \def\PYGZam{\discretionary{\char`\&}{\Wrappedafterbreak}{\char`\&}}%
            \def\PYGZlt{\discretionary{}{\Wrappedafterbreak\char`\<}{\char`\<}}%
            \def\PYGZgt{\discretionary{\char`\>}{\Wrappedafterbreak}{\char`\>}}%
            \def\PYGZsh{\discretionary{}{\Wrappedafterbreak\char`\#}{\char`\#}}%
            \def\PYGZpc{\discretionary{}{\Wrappedafterbreak\char`\%}{\char`\%}}%
            \def\PYGZdl{\discretionary{}{\Wrappedafterbreak\char`\$}{\char`\$}}%
            \def\PYGZhy{\discretionary{\char`\-}{\Wrappedafterbreak}{\char`\-}}%
            \def\PYGZsq{\discretionary{}{\Wrappedafterbreak\textquotesingle}{\textquotesingle}}%
            \def\PYGZdq{\discretionary{}{\Wrappedafterbreak\char`\"}{\char`\"}}%
            \def\PYGZti{\discretionary{\char`\~}{\Wrappedafterbreak}{\char`\~}}%
        }
        % Some characters . , ; ? ! / are not pygmentized.
        % This macro makes them "active" and they will insert potential linebreaks
        \newcommand*\Wrappedbreaksatpunct {%
            \lccode`\~`\.\lowercase{\def~}{\discretionary{\hbox{\char`\.}}{\Wrappedafterbreak}{\hbox{\char`\.}}}%
            \lccode`\~`\,\lowercase{\def~}{\discretionary{\hbox{\char`\,}}{\Wrappedafterbreak}{\hbox{\char`\,}}}%
            \lccode`\~`\;\lowercase{\def~}{\discretionary{\hbox{\char`\;}}{\Wrappedafterbreak}{\hbox{\char`\;}}}%
            \lccode`\~`\:\lowercase{\def~}{\discretionary{\hbox{\char`\:}}{\Wrappedafterbreak}{\hbox{\char`\:}}}%
            \lccode`\~`\?\lowercase{\def~}{\discretionary{\hbox{\char`\?}}{\Wrappedafterbreak}{\hbox{\char`\?}}}%
            \lccode`\~`\!\lowercase{\def~}{\discretionary{\hbox{\char`\!}}{\Wrappedafterbreak}{\hbox{\char`\!}}}%
            \lccode`\~`\/\lowercase{\def~}{\discretionary{\hbox{\char`\/}}{\Wrappedafterbreak}{\hbox{\char`\/}}}%
            \catcode`\.\active
            \catcode`\,\active
            \catcode`\;\active
            \catcode`\:\active
            \catcode`\?\active
            \catcode`\!\active
            \catcode`\/\active
            \lccode`\~`\~
        }
    \makeatother

    \let\OriginalVerbatim=\Verbatim
    \makeatletter
    \renewcommand{\Verbatim}[1][1]{%
        %\parskip\z@skip
        \sbox\Wrappedcontinuationbox {\Wrappedcontinuationsymbol}%
        \sbox\Wrappedvisiblespacebox {\FV@SetupFont\Wrappedvisiblespace}%
        \def\FancyVerbFormatLine ##1{\hsize\linewidth
            \vtop{\raggedright\hyphenpenalty\z@\exhyphenpenalty\z@
                \doublehyphendemerits\z@\finalhyphendemerits\z@
                \strut ##1\strut}%
        }%
        % If the linebreak is at a space, the latter will be displayed as visible
        % space at end of first line, and a continuation symbol starts next line.
        % Stretch/shrink are however usually zero for typewriter font.
        \def\FV@Space {%
            \nobreak\hskip\z@ plus\fontdimen3\font minus\fontdimen4\font
            \discretionary{\copy\Wrappedvisiblespacebox}{\Wrappedafterbreak}
            {\kern\fontdimen2\font}%
        }%

        % Allow breaks at special characters using \PYG... macros.
        \Wrappedbreaksatspecials
        % Breaks at punctuation characters . , ; ? ! and / need catcode=\active
        \OriginalVerbatim[#1,codes*=\Wrappedbreaksatpunct]%
    }
    \makeatother

    % Exact colors from NB
    \definecolor{incolor}{HTML}{303F9F}
    \definecolor{outcolor}{HTML}{D84315}
    \definecolor{cellborder}{HTML}{CFCFCF}
    \definecolor{cellbackground}{HTML}{F7F7F7}

    % prompt
    \makeatletter
    \newcommand{\boxspacing}{\kern\kvtcb@left@rule\kern\kvtcb@boxsep}
    \makeatother
    \newcommand{\prompt}[4]{
        {\ttfamily\llap{{\color{#2}[#3]:\hspace{3pt}#4}}\vspace{-\baselineskip}}
    }
    

    
    % Prevent overflowing lines due to hard-to-break entities
    \sloppy
    % Setup hyperref package
    \hypersetup{
      breaklinks=true,  % so long urls are correctly broken across lines
      colorlinks=true,
      urlcolor=urlcolor,
      linkcolor=linkcolor,
      citecolor=citecolor,
      }
    % Slightly bigger margins than the latex defaults
    
    \geometry{verbose,tmargin=1in,bmargin=1in,lmargin=1in,rmargin=1in}
    
    

\begin{document}
    
    \maketitle
    
    

    
    \hypertarget{poli-273}{%
\section{POLI 273}\label{poli-273}}

\hypertarget{causal-inference}{%
\subsection{Causal Inference}\label{causal-inference}}

\hypertarget{lecture-06---dags-for-causal-identification-and-regression}{%
\subsubsection{Lecture 06 - DAGs for Causal Identification and
Regression}\label{lecture-06---dags-for-causal-identification-and-regression}}

    \hypertarget{announcements}{%
\subsection{Announcements}\label{announcements}}

\begin{itemize}
\item
  This class GitHub: https://github.com/umbertomig/POLI273
\item
  PS01: How is it going?
\item
  Qualtrics Videos: It is still missing the Conjoint Video and the
  Case-Control Video.

  \begin{itemize}
  \tightlist
  \item
    I'll post them this week
  \end{itemize}
\end{itemize}

    \hypertarget{agenda-for-today}{%
\subsection{Agenda for Today}\label{agenda-for-today}}

\begin{itemize}
\tightlist
\item
  DAGs:

  \begin{itemize}
  \tightlist
  \item
    d-Separation
  \item
    Backdoor criterion
  \item
    Why they are good / connections with PO.
  \end{itemize}
\item
  Regression Analysis:

  \begin{itemize}
  \tightlist
  \item
    Linear Regression
  \item
    Consistency
  \item
    Regression for Causal Inference
  \end{itemize}
\end{itemize}

    \hypertarget{dags}{%
\subsection{DAGs}\label{dags}}

\begin{itemize}
\item
  Three relations in the paths:

  \begin{itemize}
  \tightlist
  \item
    Chains:
  \end{itemize}

  \[T \longrightarrow C \longrightarrow Y\]

  \begin{itemize}
  \tightlist
  \item
    Forks:
  \end{itemize}

  \[T \longleftarrow C \longrightarrow Y\]

  \begin{itemize}
  \tightlist
  \item
    Inverted Forks (Colliders):
  \end{itemize}

  \[T \longrightarrow C \longleftarrow Y\]
\end{itemize}

    \hypertarget{dags}{%
\subsection{DAGs}\label{dags}}

\begin{itemize}
\tightlist
\item
  Consider this DAG (Elwert, 2013):
\end{itemize}

\begin{figure}
\centering
\includegraphics{../img/dag1.png}
\caption{img}
\end{figure}

\begin{itemize}
\tightlist
\item
  Let us see what happens in the data.
\end{itemize}

    \hypertarget{dags}{%
\subsection{DAGs}\label{dags}}

\begin{itemize}
\item
  {[}Pearl 1988{]} \textbf{d-Separation}: A path between two variables
  \(A\) and \(B\) is said to be d-separated (blocked; closed) if:

  \begin{enumerate}
  \def\labelenumi{\arabic{enumi}.}
  \tightlist
  \item
    The path contains a \textbf{non-collider that has been conditioned
    on}, OR
  \item
    The path contains a \textbf{collider that has \emph{not} been
    conditioned on}.
  \end{enumerate}
\item
  \begin{enumerate}
  \def\labelenumi{\arabic{enumi}.}
  \setcounter{enumi}{1}
  \tightlist
  \item
    If two variables \(A\) and \(B\) are d-separated conditional on a
    third variable (or sets of variables) \(C\), then \(A \perp B | C\).
  \item
    If your DAG is faithful and the variables \(A\) and \(B\) are
    d-connected (or not d-separated), then they are dependent
    (faithfulness: assumes your DAG is certain; weak faithfulness:
    assume your DAG is a conjecture).
  \end{enumerate}
\end{itemize}

    \hypertarget{dags}{%
\subsection{DAGs}\label{dags}}

\begin{itemize}
\tightlist
\item
  Are \(X\) and \(Y\) d-separated?
\end{itemize}

\begin{figure}
\centering
\includegraphics{../img/dag3.png}
\caption{img}
\end{figure}

\begin{enumerate}
\def\labelenumi{\arabic{enumi}.}
\tightlist
\item
  Draw the paths.
\item
  Study them.
\item
  What happens when you start conditioning?
\end{enumerate}

    \hypertarget{dags}{%
\subsection{DAGs}\label{dags}}

\begin{itemize}
\tightlist
\item
  {[}Shpister et al.~2010{]}: \textbf{Adjustment criterion}: A set of
  observed variables \(Z\) (that may be empty) satisfies the adjustment
  criterion relative to the total causal effect of a treatment \(T\) on
  an outcome \(Y\) if:

  \begin{enumerate}
  \def\labelenumi{\arabic{enumi}.}
  \tightlist
  \item
    \(Z\) blocks all non-causal paths from \(Z\) to \(Y\), AND
  \item
    No variable in \(Z\) lies on or decents from a causal path from
    \(T\) to \(Y\).
  \end{enumerate}
\end{itemize}

    \hypertarget{dags}{%
\subsection{DAGs}\label{dags}}

Consider the following DAG, where we are interested in the effect of a
treatment \(T\) on \(Y\):

\begin{figure}
\centering
\includegraphics{../img/dag6.png}
\caption{img}
\end{figure}

\begin{enumerate}
\def\labelenumi{\arabic{enumi}.}
\tightlist
\item
  Draw the paths.
\item
  Study them.
\item
  Can we use the adjustment criterion to identify the effects causally?
  (Hint: Nine possible adjustments. Find two).
\end{enumerate}

    \hypertarget{dags}{%
\subsection{DAGs}\label{dags}}

\begin{itemize}
\tightlist
\item
\item
  {[}Pearl 1993{]}: \textbf{Backdoor criterion}: A set of observed
  variables \(Z\) (that may be empty) satisfies the backdoor criterion
  relative to the total causal effect of a treatment \(T\) on an outcome
  \(Y\) if:

  \begin{enumerate}
  \def\labelenumi{\arabic{enumi}.}
  \tightlist
  \item
    No element of \(Z\) is a descendant of \(T\), AND
  \item
    \(Z\) blocks all backdoor paths from \(T\) to \(Y\).
  \end{enumerate}
\end{itemize}

    \hypertarget{dags}{%
\subsection{DAGs}\label{dags}}

Consider the following DAG, where we are interested in the effect of a
treatment \(T\) on \(Y\):

\begin{figure}
\centering
\includegraphics{../img/dag7.png}
\caption{img}
\end{figure}

\begin{enumerate}
\def\labelenumi{\arabic{enumi}.}
\tightlist
\item
  Draw the paths.
\item
  Study them.
\item
  Can we use the \emph{backdoor criterion} to identify the effect
  causally? (Met by seven adjustments. Find two.)
\end{enumerate}

    \hypertarget{dags}{%
\subsection{DAGs}\label{dags}}

\begin{itemize}
\item
  More on DAGs later. Now, let us see the implication for Experimental
  Political Science.
\item
  Suppose we have a random treatment \(Z\) and an outcome of interest
  \(Y\). And let us assume that:
\end{itemize}

\[ (Y_{1}, Y_{0}) \perp Z \ \text{ and positivity}\]

This means that:

\begin{itemize}
\tightlist
\item
  \(Y_1 \perp Z\) and \(Y_0 \perp Z\). From what we know from the
  adjustment criterion:
\end{itemize}

\[ \tau = \mathbb{E}\big[Y_{1} - Y_{0}\big] =  \mathbb{E}\big[Y_{1}\big] -  \mathbb{E}\big[Y_{0}\big] \]

\begin{figure}
\centering
\includegraphics{../img/dag4.png}
\caption{img}
\end{figure}

    \hypertarget{dags}{%
\subsection{DAGs}\label{dags}}

\begin{itemize}
\tightlist
\item
  Suppose we have a treatment \(Z\), a variable \(X\), and an outcome of
  interest \(Y\). And let us assume that:
\end{itemize}

\[ (Y_{1}, Y_{0}) \perp Z | X \ \text{ and positivity}\]

\begin{itemize}
\tightlist
\item
  This means that:
\end{itemize}

\begin{enumerate}
\def\labelenumi{\arabic{enumi}.}
\tightlist
\item
  \(\mathbb{E}\big[Y_1\big] = \sum_{\text{Supp}(X)} \mathbb{E}[Y|Z = 1, X = x]\mathbb{P}(X = x)\)
\item
  \(\mathbb{E}\big[Y_0\big] = \sum_{\text{Supp}(X)} \mathbb{E}[Y|Z = 0, X = x]\mathbb{P}(X = x)\)
\end{enumerate}

And:

\[ \tau = \mathbb{E}\big[Y_{1} - Y_{0}\big] =  \mathbb{E}\big[Y_{1}\big] -  \mathbb{E}\big[Y_{0}\big] \]

\begin{figure}
\centering
\includegraphics{../img/dag5.png}
\caption{img}
\end{figure}

    \hypertarget{regression-for-causal-identification}{%
\section{Regression for Causal
Identification}\label{regression-for-causal-identification}}

    \hypertarget{regression-for-causal-identification}{%
\subsection{Regression for Causal
Identification}\label{regression-for-causal-identification}}

\begin{itemize}
\item
  A linear regression is a \emph{special} projection on a space.
\item
  Suppose we have two sets of variables \((X, y)\) (\(X\) can be more
  than one var).
\item
  The conditional expectation of \(y\) given \(X\) can be decompose as:
\end{itemize}

\[ \bf{y} = \mathbb{E}\big[\bf{y} \ | \ \bf{X}\big] + \bf{\varepsilon} \]

    \hypertarget{regression-for-causal-identification}{%
\subsection{Regression for Causal
Identification}\label{regression-for-causal-identification}}

\[ \bf{y} = \mathbb{E}\big[\bf{y} \ | \ \bf{X}\big] + \bf{\varepsilon} \]

And if this is the case, then:

\textbf{Theorem:}
\(\mathbb{E}\big[\bf{\varepsilon} \ | \ \bf{X} \big] = \bf{0}\).
Moreover:

\begin{enumerate}
\def\labelenumi{\arabic{enumi}.}
\item
  For any function \(\phi\),
  \(\mathbb{E}\big[\phi(\bf{X})\bf{\varepsilon} \big] = \bf{0}\).
\item
  In particular: \$\mathbb{E}\big[\bf{X}'\bf{\varepsilon} \big] =
  \textbackslash bf\{0\} \$
\end{enumerate}

    \hypertarget{regression-for-causal-identification}{%
\subsection{Regression for Causal
Identification}\label{regression-for-causal-identification}}

\begin{itemize}
\tightlist
\item
  Now, pick the result:
\end{itemize}

\[\mathbb{E}(\bf{X}'\varepsilon) = 0\]

\begin{itemize}
\tightlist
\item
  What does it mean?

  \begin{itemize}
  \tightlist
  \item
    In linear-algebra-terms, it means that if you project the variables
    on the residuals, the relationship is \emph{orthogonal}.
  \item
    Remember that, by the cosine law: the angle between two vectors is
    \(cos(\theta) = \dfrac{\bf{a}'\bf{b}}{\big|\bf{a}\big| \ \big|\bf{b}\big|}\).
  \end{itemize}
\end{itemize}

    \hypertarget{regression-for-causal-identification}{%
\subsection{Regression for Causal
Identification}\label{regression-for-causal-identification}}

\begin{itemize}
\tightlist
\item
  Or:
\end{itemize}

\begin{figure}
\centering
\includegraphics{../img/regasproj.png}
\caption{img}
\end{figure}

    \hypertarget{regression-for-causal-identification}{%
\subsection{Regression for Causal
Identification}\label{regression-for-causal-identification}}

Let \(f(\bf{X}) = \bf{X}'\) and that the conditional expectation of
\(y\) given \(X\) is linear (i.e.,
\$\mathbb{E}(\bf{y}|\bf{X}) = \bf{X}\textbackslash bf\{\beta\} \$).
Then:

\[
\begin{align}
\bf{0} & = \mathbb{E}(f(\bf{X})\bf{\varepsilon}) \\
        & = \mathbb{E}(\bf{X}'\bf{\varepsilon}) \\
        & = \mathbb{E}(\bf{X}'(\bf{y} - \mathbb{E}(\bf{y}|\bf{X}))) \\
        & = \mathbb{E}(\bf{X}'(\bf{y} - \bf{X}\bf{\beta})) \\
        & = \mathbb{E}(\bf{X}'\bf{y}) - \mathbb{E}(\bf{X}'\bf{X}\bf{\beta})
\end{align}
\]

Thus:

\[ \bf{\beta} \ = \ \mathbb{E}\big(\bf{X}'\bf{X}\big)^{-1}\mathbb{E}\big(\bf{X}'\bf{y}\big) \]

    \hypertarget{regression-for-causal-identification}{%
\subsection{Regression for Causal
Identification}\label{regression-for-causal-identification}}

\begin{itemize}
\tightlist
\item
  Let us now look at each individual value.
\end{itemize}

\textbf{Theorem}: Assuming that for each \(i\):

\[ Y_i \ = \ \mathbb{E}(Y_i|X_i) + \varepsilon_i \]

Then, \(\mathbb{E}\big[\varepsilon_i|X_i\big] = 0\).

    \hypertarget{regression-for-causal-identification}{%
\subsection{Regression for Causal
Identification}\label{regression-for-causal-identification}}

What is the \texttt{minimum\ mean\ squared\ error\ predictor} (MHE,
Chapter 3)?

\textbf{Theorem}: Suppose that there exists a function \(m(X_i)\) that
minimizes:

\[ S(m(X_i)) \ = \ \mathbb{E}\bigg[\big(Y_i - m(X_i)\big)^2\bigg] \]

Then, \(m(X_i) = \mathbb{E}\big(Y_i|X_i\big)\).

    \hypertarget{regression-for-causal-identification}{%
\subsection{Regression for Causal
Identification}\label{regression-for-causal-identification}}

\begin{itemize}
\tightlist
\item
  Since we operate in a sample, can we recover the \emph{population}
  parameters using the \emph{sample} parameters?
\end{itemize}

\textbf{Theorem}: Let
\$\bf{\beta} \ = \ \mathbb{E}\big[\bf{X}'\bf{X}\big]^{-1}\mathbb{E}\big[\bf{X}'\bf{y}\big] $ and let the MSE estimator in the sample $\widehat{\bf{\beta}} \ = \ \big[\bf{X}'\bf{X}\big]^{-1}\bf{X}'\textbackslash bf\{y\}
\$. Then:

\[ \mathbb{E}\big[\widehat{\bf{\beta}} | \bf{X}\big] \ = \ \bf{\beta} + \mathbb{E}\bigg[ \big[\bf{X}'\bf{X}\big]^{-1}\bf{X}'\bf{\varepsilon} \ | \ \bf{X}\bigg] \ = \ \bf{\beta}\]

But how does that help us to find the ATE?

    \hypertarget{regression-for-causal-identification}{%
\subsection{Regression for Causal
Identification}\label{regression-for-causal-identification}}

Let the ATE:

\[ \tau \ = \ \mathbb{E}\big[Y_{1i} - Y_{0i}\big] \]

And assume all assumptions hold: positivity + CIA. This means that
\((Y_1, Y_0) \perp Z\) (and remember that \(Y_{1i} = (Y_i | Z_i = 1)\)).

    \hypertarget{regression-for-causal-identification}{%
\subsection{Regression for Causal
Identification}\label{regression-for-causal-identification}}

And let us consider that \(Y_i\) is equal to:

\[ Y_i \ = \ Z_i \beta + \varepsilon_i \]

It is easy to see that:

\[ \mathbb{E} \big[Y_i | Z_i \big] \ = \ \mathbb{E} \big[Z_i | Z_i\big] \beta + \mathbb{E} \big[\varepsilon_i  | Z_i \big] \]

And since these are exogenous:
\(\mathbb{E} \big[\varepsilon_i | Z_i \big] = 0\), what makes:

\[ \beta \ = \ \mathbb{E} \big[Y_i | Z_i \big] \]

    \hypertarget{regression-for-causal-identification}{%
\subsection{Regression for Causal
Identification}\label{regression-for-causal-identification}}

This is still not that helpful, but remember that
\(Y_i = Z_iY_{1i} + (1 - Z_i)Y_{0i}\). A little algebra gets:

\[ Y_i = Y_{0i} + (Y_{1i} - Y_{0i})Z_i \]

And plugging this in:

\[
\begin{align}
\beta & = \mathbb{E} \big[Y_{0i} + (Y_{1i} - Y_{0i})Z_i | Z_i \big]
\end{align}
\]

And now what? Complete the proof to find that: \(\beta = \tau\)!

    \hypertarget{regression-for-causal-identification}{%
\subsection{Regression for Causal
Identification}\label{regression-for-causal-identification}}

\hypertarget{the-linear-regression-coefficient-is-equivalent-to-the-differences-in-means-estimator.}{%
\subsubsection{The linear regression coefficient is equivalent to the
differences-in-means
estimator.}\label{the-linear-regression-coefficient-is-equivalent-to-the-differences-in-means-estimator.}}

    \hypertarget{questions}{%
\subsection{Questions?}\label{questions}}

    \hypertarget{see-you-in-the-next-class}{%
\subsection{See you in the next
class!}\label{see-you-in-the-next-class}}


    % Add a bibliography block to the postdoc
    
    
    
\end{document}
